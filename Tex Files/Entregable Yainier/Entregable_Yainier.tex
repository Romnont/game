\documentclass[12pt,letterpaper]{report}

% PAQUETES UTILIZADOS
\usepackage[spanish]{babel}
\usepackage{cite}
\usepackage[utf8]{inputenc}
\usepackage{graphicx}
\usepackage{mathrsfs}
\usepackage{enumitem}
\usepackage{float}
\usepackage{researchdiary_png}

\newcommand{\workingDate}{\textsc{ }}
\newcommand{\userName}{ }
\newcommand{\institution}{ }



%////////////// Inicio del documento //////////////

\begin{document}

\begin{titlepage}
\centering
%////////////// PORTADA //////////////
\begin{center}
\includegraphics[scale=0.5]{uaem_logo.png}
\end{center}
\vspace{0.5cm}
{\bfseries\normalsize UNIVERSIDAD AUTÓNOMA DEL ESTADO DE MORELOS\newline
INSTITUTO DE INVESTIGACIÓN EN CIENCIAS BÁSICAS Y APLICADAS\newline 	
CENTRO DE INVESTIGACIÓN EN CIENCIAS
 \par}
\vspace{1.5cm}
{\scshape\Large MS-2k1 \par}
\vspace{1.5cm}
{\bfseries\scshape\Huge REPORTE FINAL DE \par}
\vspace{0.3cm}
{\Large{INTRODUCCIÓN A LA PROGRAMACIÓN DE VIDEOJUEGOS EN C++ Y PYTHON}}\\
\vspace{0.3cm}
\vspace{1cm}
{\Large PRESENTA \\}
\vspace{0.3cm}
{\bfseries\scshape\Large\textit{ANDRÉS REYES - FRANCISCO AQUINO - EDUARDO NUÑEZ}\par}
\vspace{1cm}
\Large PROFESOR: \\
\itshape\Large Mtro. Yainier Labrada Nueva \par

\end{titlepage}

%////////////// TABLA DE CONTENIDOS //////////////
\tableofcontents
%====================================================================================
%
%
% CAPITULO 1: INTRODUCCION
%
%
%
\chapter{Introducción} %Capitulos

En los últimos años, la industria de los videojuegos se ha visto envuelta en una revolución tecnológica abismal, el poder de computo y las ascendentes tecnologías en el área de la comunicación digital, han abierto puertas inmensas para todas las ramas de tecnología en el mundo, entre ellas, el desarrollo de videojuegos.

Los videojuegos recrean entornos y situaciones virtuales en los que la persona puede controlar a uno o mas personajes de este entorno. 

Este proyecto tiene el enfoque de poder aprender y analizar la programación de videojuegos, que emplea muchos paradigmas de la programación como la programación orientada a objetos, a eventos, procesos, y muchos conceptos más. 

%====================================================================================
%
%
% CAPITULO 1: INTRODUCCION Y OBJETIVOS
%
%
%

\chapter{Objetivos}

\section{Objetivos específicos}

Esta sección esta enfocada a las distintas metas que se propusieron alcanzar como base para la realización de este proyecto. Las siguientes son:

\begin{itemize}

\item Adquirir las bases y el conocimiento para poder desarrollar un videojuego desde cero.

\item Comprender como se emplea la programación y la lógica en algo tan complejo como lo es un videojuego.

\item Entender la modelación y aplicación de la programación orientada a objetos en un sistema en tiempo real, con eventos y objetos en constante cambio, dentro de un entorno simulado.

\end{itemize}


\section{Objetivos generales}

Los objetivos enlistados aquí, son metas que sobre la marcha, se vislumbraban completables a lo largo del desarollo del videojuego. Algunas de ellas son:

\begin{itemize}

\item Aprender y reforzar un lenguaje de programación de alta popularidad como lo es Python.

\item Aplicar y desarrollar conceptos adquiridos en la formación académica en un proyecto real e integral.

\item Desarrollar y pulir las distintas habilidades de programación que se han adquirido a lo largo de la formación académica.


\end{itemize}


%====================================================================================
%
%
%
% CAPITULO 2: MARCO TEORICO
%
%
%



\chapter{Marco teórico}

\section{Anaconda}




\section{Python}

Python es un lenguaje de programación de alto nivel, muy sencillo de aprender a utilizar, y además es multiparadigma, por lo que python se presta para realizar cualquier tipo de tarea, como es el caso: un videojuego.

Es un lenguaje de programación versátil, utilizado en muchas áreas, como lo son la estadística, la comunicación y la ciencia de datos. 

Algunas de sus ventajas son:

\begin{itemize}

\item Es simple y rapido: puedes hacer mucho con pocas lineas de código.

\item Flexible: No necesariamente vas a tener que preocuparte por detalles de tipificación y declaraciones.

\item Multiplataforma: Se puede ejecutar en cualquier sistema operativo y cuenta con librerías especificas para cada uno de estos.

\item Open Source: Su comunidad se encarga de mantener al lenguaje y crear todos sus recursos de forma gratuita.


\end{itemize}



\section{Pygame}

Pygame es un conjunto de módulos del lenguaje Python que permiten y facilitan la creación de videojuegos en dos dimensiones de una manera sencilla, gracias al manejo de sprites.





\chapter{Composición y desarrollo}




\chapter{Resultado}




\chapter{Conclusiones}


\begin{thebibliography}{0}

  \bibitem{AirSim2017} Shah, S., Dey, D., Lovett, C., \& Kapoor, A. (2018). Airsim: High-fidelity visual and physical simulation for autonomous vehicles. In Field and service robotics (pp. 621-635). Springer, Cham.
  
\end{thebibliography}

\end{document}